% !TEX TS-program = pdflatex
% !TEX encoding = UTF-8 Unicode

% This is a simple template for a LaTeX document using the "article" class.
% See "book", "report", "letter" for other types of document.

\documentclass[8pt]{article} % use larger type; default would be 10pt

\usepackage[utf8]{inputenc} % set input encoding (not needed with XeLaTeX)
\usepackage{bchart}
\usepackage{longtable}
\usepackage{pgfgantt}
\usepackage{calendar} % Use the calendar.sty style
\usepackage{calc}
\usepackage{ifthen}
\usepackage{tkz-base}
\usepackage{tikz}
\usepackage{hyperref}
\usepackage{tkz-kiviat,numprint,fullpage} 
\usepackage{pgfplotstable} 
\usepackage{pdfpages} 
\usepackage{tikz-3dplot}
\usepackage{pgfgantt}
\usetikzlibrary{arrows}
\RequirePackage{pgfcalendar}
\RequirePackage{tikz}
\usetikzlibrary{%
arrows, backgrounds, calc,%
patterns, positioning, shapes.geometric%
}
%%% Examples of Article customizations
% These packages are optional, depending whether you want the features they provide.
% See the LaTeX Companion or other references for full information.

\usepackage{textcomp}
%\usepackage{hyperref}

%%% PAGE DIMENSIONS
\usepackage{geometry} % to change the page dimensions
\geometry{a4paper} % or letterpaper (US) or a5paper or....
% \geometry{margin=2in} % for example, change the margins to 2 inches all round
% \geometry{landscape} % set up the page for landscape
%   read geometry.pdf for detailed page layout information

\usepackage{graphicx} % support the \includegraphics command and options

% \usepackage[parfill]{parskip} % Activate to begin paragraphs with an empty line rather than an indent

%%% PACKAGES
\usepackage{booktabs} % for much better looking tables
\usepackage{array} % for better arrays (eg matrices) in maths
\usepackage{paralist} % very flexible & customisable lists (eg. enumerate/itemize, etc.)
\usepackage{verbatim} % adds environment for commenting out blocks of text & for better verbatim
\usepackage{subfig} % make it possible to include more than one captioned figure/table in a single float
% These packages are all incorporated in the memoir class to one degree or another...

%%% HEADERS & FOOTERS
\usepackage{fancyhdr} % This should be set AFTER setting up the page geometry
\pagestyle{fancy} % options: empty , plain , fancy
\renewcommand{\headrulewidth}{0pt} % customise the layout...
\lhead{}\chead{}\rhead{}
\lfoot{}\cfoot{\thepage}\rfoot{}

%%% SECTION TITLE APPEARANCE
\usepackage{sectsty}
\allsectionsfont{\sffamily\mdseries\upshape} % (See the fntguide.pdf for font help)
% (This matches ConTeXt defaults)

%%% ToC (table of contents) APPEARANCE
\usepackage[nottoc,notlof,notlot]{tocbibind} % Put the bibliography in the ToC
\usepackage[titles,subfigure]{tocloft} % Alter the style of the Table of Contents
\renewcommand{\cftsecfont}{\rmfamily\mdseries\upshape}
\renewcommand{\cftsecpagefont}{\rmfamily\mdseries\upshape} % No bold!

%%% END Article customizations

%%% The "real" document content comes below...

\title{Project management}
\author{\copyright Frederic Kerdraon}
%\date{} % Activate to display a given date or no date (if empty),
         % otherwise the current date is printed 

\begin{document}
\maketitle
\tableofcontents

\section{Introduction}

This document summurizes all the important informations necessary to facilitate things and remove a lot of stress. It's been put together thanks to \LaTeX. This is designed to help make optimal decisions for a not so short lifetime.

%{\footnote
%Ce n'est pas parceque les choses sont difficiles que nous n'osons pas, c'est parceque nous n'osons pas qu'elles sont difficiles
%}
%{\footnote
%For once you have tasted flight you will walk the earth with your eyes turned skywards, 
%for there you have been and there you will long to return.
%Leonardo da Vinci
%
%}
%\makebox[2\width]{hello}

\newcounter{a}
\newcounter{b}

%----------------------------------------------------------
\newcommand{\slice}[4]{
  \pgfmathparse{0.5*#1+0.5*#2}
  \let\midangle\pgfmathresult

   slice
  \draw[thick,fill=black!10] (0,0) -- (#1:1) arc (#1:#2:1) -- cycle;

   outer label
  \node[label=\midangle:#4] at (\midangle:1) {};

   inner label
  \pgfmathparse{min((#2-#1-10)/110*(-0.3),0)}
  \let\temp\pgfmathresult
  \pgfmathparse{max(\temp,-0.5) + 0.8}
  \let\innerpos\pgfmathresult
  \node at (\midangle:\innerpos) {#3};
}

\section{Management summary}

Before starting a new project, you'd better find some supporters...
\begin{figure}[ht!]
\centering
\includegraphics[width=90mm]{World3.png}
\caption{Connections \label{overflow}}
\end{figure}

%{\footnote
%En gros tous les dossiers que j'ai au bureau....;-)
%}


\subsection{Summary}

\subsubsection{The logo for the Project management}
%\includegraphics[width=120mm]{SeismicSphere.pdf}
%\includepdf{SeismicSphere.pdf}
\input{SeismicSphere}

%\subsubsection{A decision tree}
%\includegraphics{Multiplot.png}
%\includepdf{MultiPlot.pdf}

%\subsubsection{Multiplot}
%\includegraphics[width=120mm]{SeismicSphere.pdf}
%\includepdf{MultiPlot.pdf}

%\subsubsection{Scatter2}
%\includegraphics[width=120mm]{SeismicSphere.pdf}
%\includepdf{Scatter2.pdf}

%\subsubsection{Regressoin}
%\includegraphics[width=120mm]{SeismicSphere.pdf}
%\includepdf{Regression.pdf}

%\subsubsection{The Workflow}
%\includegraphics[width=120mm]{SeismicSphere.pdf}
%Removed because it's UGLY\\
%\includepdf{RuleBasedTree.pdf}

\subsubsection{The green side first !}
Tasks that we couldnt complete in time\\
Highest investment or Budget (time and monney)\\
Resources that deserve hols\\
Tasks that we dont want to fuck up (critical path of the highest investment)\\

\subsubsection{The new challenges ahead !}
Highest investment or Budget (time and monney)\\
Tasks that we couldnt complete in time\\
Resources that need to move their asses\\
Tasks that we dont want to fuck up (critical path of the highest investment)\\

\subsection{Events}
From events (Meetings, Deliverables)
\begin{longtable}{|c|c|c|c|}
\hline
\multicolumn{4}{|c|}{Events} \\
\hline
Date & Type & Name & Template\\
\hline
0000-00-00 00:00:00 & Deliverable & Inception & Inception.tex\\
\hline
0000-00-00 00:00:00 & Deliverable & Specification & Specification.tex\\
\hline
0000-00-00 00:00:00 & Deliverable & Clearchoice & Clearchoice.tex\\
\hline
0000-00-00 00:00:00 & Deliverable & External design & External design.tex\\
\hline
0000-00-00 00:00:00 & Deliverable & Internal design & Internal design.tex\\
\hline
0000-00-00 00:00:00 & Deliverable & Test documentation & Test documentation.tex\\
\hline
0000-00-00 00:00:00 & Deliverable & Release notes & Release notes.tex\\
\hline
0000-00-00 00:00:00 & Deliverable & Post implementation review & Post implementation revie\\
\hline
0000-00-00 00:00:00 & Deliverable & Support documentation & Support documentation.tex\\
\hline
0000-00-00 00:00:00 & Deliverable & Inception & Inception.tex\\
\hline
0000-00-00 00:00:00 & Deliverable & Specification & Specification.tex\\
\hline
0000-00-00 00:00:00 & Deliverable & Clearchoice & Clearchoice.tex\\
\hline
0000-00-00 00:00:00 & Deliverable & External design & External design.tex\\
\hline
0000-00-00 00:00:00 & Deliverable & Internal design & Internal design.tex\\
\hline
0000-00-00 00:00:00 & Deliverable & Test documentation & Test documentation.tex\\
\hline
0000-00-00 00:00:00 & Deliverable & Release notes & Release notes.tex\\
\hline
0000-00-00 00:00:00 & Deliverable & Post implementation review & Post implementation revie\\
\hline
0000-00-00 00:00:00 & Deliverable & Support documentation & Support documentation.tex\\
\hline
0000-00-00 00:00:00 & Birthday & Annif Antoine & Annif Antoine.tex\\
\hline
0000-00-00 00:00:00 & Meeting & Council & Council.tex\\
\hline
0000-00-00 00:00:00 & Meeting & Council & Council.tex\\
\hline
0000-00-00 00:00:00 & Daily & Daily setup & ManagementSummary.pdf\\
\hline
0000-00-00 00:00:00 & Daily & Daily setup & ManagementSummary.pdf\\
\hline
0000-00-00 00:00:00 & Event & Renewal Ins Auto & Letter.pdf\\
\hline
0000-00-00 00:00:00 & Event & Renewal Ins Bateau & Letter.pdf\\
\hline
0000-00-00 00:00:00 & Event & Renewal Ins Maison & Letter.pdf\\
\hline
0000-00-00 00:00:00 & Event & Car servicing & Letter.pdf\\
\hline
0000-00-00 00:00:00 & Event & Renewal paiment card & Letter.pdf\\
\hline
0000-00-00 00:00:00 & Event & Taxes declaration & Letter.pdf\\
\hline
0000-00-00 00:00:00 & Event & Taxes paiement & Letter.pdf\\
\hline
0000-00-00 00:00:00 & Event & Christmas tree Apologic & Letter.pdf\\
\hline
0000-00-00 00:00:00 & Event & Birthday me & Birthday me.tex\\
\hline
0000-00-00 00:00:00 & Event & Birthday Juliette & Birthday Juliette.tex\\
\hline
0000-00-00 00:00:00 & Event & Birthday Thibault & Birthday Thibault.tex\\
\hline
0000-00-00 00:00:00 & Event & Birthday Emilie & Birthday Emilie.tex\\
\hline
0000-00-00 00:00:00 & Event & Birthday Antoine & Birthday Antoine.tex\\
\hline
0000-00-00 00:00:00 & Event & Poker tour St Meen & Poker tour St Meen.tex\\
\hline
0000-00-00 00:00:00 & Event & Clean the deck & Clean the deck.tex\\
\hline
0000-00-00 00:00:00 & Event & Clean the deck & Clean the deck.tex\\
\hline
0000-00-00 00:00:00 & Event & Clean the deck & Clean the deck.tex\\
\hline
0000-00-00 00:00:00 & Event & Clean the deck & Clean the deck.tex\\
\hline
0000-00-00 00:00:00 & Event & Clean the deck & Clean the deck.tex\\
\hline
0000-00-00 00:00:00 & Event & Clean the deck & Clean the deck.tex\\
\hline
 ... & ... & ... & ... \\
\hline
\hline
\end{longtable}


\subsubsection{Tasks aggregated}
Ajouter time to expiry, Feasible time, Feasible Budget, Feasible Complexity/Risks
RAF, Done, Completion perc, Maturity\\
\begin{longtable}{|c|c|c|}
\hline
\multicolumn{3}{|c|}{Tasks} \\
\hline
Project & ROI & Complexity \\
\hline
Finance & 35082 & 12\\
\hline
Friends & 12601 & 30\\
\hline
Work & 500 & 4\\
\hline
Boat & 332 & 269\\
\hline
Day to day & 2 & 30\\
\hline
Done & 1 & 2\\
\hline
Car & 0 & 3\\
\hline
Plijadur & 0 & 3\\
\hline
Pending & 0 & 1\\
\hline
 & 0 & 1\\
\hline
\end{longtable}

\begin{bchart}[min=0,max=350,step=70,unit=k\texteuro]
\bcbar[label=MyNew project]{100000}\\
\smallskip
\bcbar[label=Admin]{1105}\\
\smallskip
\bcbar[label=Climate camp]{300}\\
\smallskip
\bcbar[label=Boat]{156}\\
\smallskip
\bcbar[label=Friends]{126}\\
\smallskip
\bcbar[label=CaMarchee]{32}\\
\smallskip
\bcbar[label=IT]{10}\\
\smallskip
\bcbar[label=Work]{10}\\
\smallskip
\bcbar[label=Guitar]{1}\\
\smallskip
\bcbar[label=Finance]{0}\\
\smallskip
\end{bchart}


%\subsubsection{Tasks 3D burndown}
%\input{3dTasks}
\subsubsection{Tasks 3D burndown}
%\includegraphics[width=1\textwidth]{Project.png}
%\input{Scatter2}
\includegraphics[width=90mm]{Scilab-burndown.png}

\subsubsection{Tasks detailed}
\begin{longtable}{|c|c|c|c|c|c|}
\hline
\multicolumn{6}{|c|}{Tasks} \\
\hline
Project & Task & Return & Cost & R/C & NumDays \\
\hline
Finance & Vendre Plijadur & 27000 & 1 & 27000 & 10\\
\hline
Finance & Vendre voiture & 8000 & 1 & 8000 & 10\\
\hline
Friends & Bouffe Armelle & 3200 & 5 & 640 & 10\\
\hline
Friends & Bouffe Christophe & 1200 & 2 & 600 & 10\\
\hline
Work & Rencontrer mecs tyfab & 500 & 1 & 500 & 10\\
\hline
Friends & Bouffe Etienne & 4000 & 10 & 400 & 10\\
\hline
Friends & Bouffe Zaz & 3800 & 10 & 380 & 10\\
\hline
Friends & Bouffe Jab & 400 & 2 & 200 & 10\\
\hline
Boat & revision moteur & 100 & 1 & 100 & 10\\
\hline
Boat & Carenner le Boat & 90 & 1 & 90 & 10\\
\hline
Finance & Faire le virement HSBC & 80 & 1 & 80 & 10\\
\hline
Boat & genois - crowd funding? & 50 & 1 & 50 & 10\\
\hline
\end{longtable}

\begin{bchart}[min=0,max=50,step=10,unit=k\texteuro]
\bcbar[label=Sell Plijadur]{15}\\
\smallskip
\bcbar[label=Qt - add global update for tasks from Qt]{1}\\
\smallskip
\bcbar[label=visiter les terrains et poser des jalons]{0}\\
\smallskip
\bcbar[label=Diner Armelle]{3}\\
\smallskip
\bcbar[label=Diner Christophe]{1}\\
\smallskip
\bcbar[label=Meet guys fromTyfab]{0}\\
\smallskip
\bcbar[label=Meet guys fromTyfab]{0}\\
\smallskip
\bcbar[label=D�jeuner Karel]{4}\\
\smallskip
\bcbar[label=Diner Zaz]{3}\\
\smallskip
\bcbar[label=Diner Jab]{0}\\
\smallskip
\bcbar[label=Sale genoa36]{0}\\
\smallskip
\bcbar[label=Motor servicing]{0}\\
\smallskip
\bcbar[label=Careen the Boat]{0}\\
\smallskip
\bcbar[label=HSBC transfer]{0}\\
\smallskip
\bcbar[label=Genoa - crowd funding?]{0}\\
\smallskip
\bcbar[label=Plastify maps]{0}\\
\smallskip
\bcbar[label=Plastify maps]{0}\\
\smallskip
\bcbar[label=Buy masks]{0}\\
\smallskip
\bcbar[label=Write log book - for history]{0}\\
\smallskip
\bcbar[label=Copy all the movies from toto/film]{0}\\
\smallskip
\bcbar[label=Copy all the movies from toto/film]{0}\\
\smallskip
\bcbar[label=Build workbench2]{0}\\
\smallskip
\bcbar[label=Copy all the movies from toto/Big bang]{0}\\
\smallskip
\bcbar[label=Copy all the movies from toto/Big bang]{0}\\
\smallskip
\bcbar[label=Change the strings]{0}\\
\smallskip
\bcbar[label=Install the GPS]{0}\\
\smallskip
\bcbar[label=Prof de guitare]{0}\\
\smallskip
\bcbar[label=Prof de guitare]{0}\\
\smallskip
\bcbar[label=Chang lamps]{0}\\
\smallskip
\bcbar[label=Clean the freezer]{0}\\
\smallskip
\bcbar[label=Check weird bank transfers]{0}\\
\smallskip
\bcbar[label=Remove radio from the boat]{0}\\
\smallskip
\bcbar[label=Contact EDF]{0}\\
\smallskip
\bcbar[label=Call Phil]{0}\\
\smallskip
\bcbar[label=Buy lighter gaz]{0}\\
\smallskip
\bcbar[label=Change the strings]{0}\\
\smallskip
\bcbar[label=Install a line on the fishing rod]{0}\\
\smallskip
\bcbar[label=Clean the deck]{0}\\
\smallskip
\bcbar[label=Grease the helm]{0}\\
\smallskip
\bcbar[label=Clean the sofas]{0}\\
\smallskip
\bcbar[label=Clean the inox]{0}\\
\smallskip
\bcbar[label=Check the anchorage]{0}\\
\smallskip
\bcbar[label=Install sound]{0}\\
\smallskip
\bcbar[label=Check battery connexions]{0}\\
\smallskip
\bcbar[label=Fix the 12v on the rack]{0}\\
\smallskip
\bcbar[label=Change camping stove support]{0}\\
\smallskip
\bcbar[label=Varnish the woods]{0}\\
\smallskip
\bcbar[label=Data - Check the numbers]{0}\\
\smallskip
\bcbar[label=Check references]{0}\\
\smallskip
\bcbar[label=Contact Jean Mich]{0}\\
\smallskip
\bcbar[label=Contacter Laurent Le bras]{0}\\
\smallskip
\bcbar[label=Go horse riding]{0}\\
\smallskip
\bcbar[label=Qt - Add combo box category for tasks]{0}\\
\smallskip
\bcbar[label=Find a badminton spot]{0}\\
\smallskip
\bcbar[label=Copy researchwork to office desktop]{0}\\
\smallskip
\bcbar[label=Qt - plug the graphs (all of them)]{0}\\
\smallskip
\bcbar[label=Scanner]{0}\\
\smallskip
\bcbar[label=Multiplex sound]{0}\\
\smallskip
\bcbar[label=Mysql - Check data ]{0}\\
\smallskip
\bcbar[label=Mysql - Generate Ids automatically]{0}\\
\smallskip
\bcbar[label=Qt - add update for tasks]{0}\\
\smallskip
\bcbar[label=Qt - add delete row tasks]{0}\\
\smallskip
\bcbar[label=Implement Clips]{0}\\
\smallskip
\bcbar[label=Implement Android (On the laptop)]{0}\\
\smallskip
\bcbar[label=Perl - Get global variables]{0}\\
\smallskip
\bcbar[label=Find protection N95]{0}\\
\smallskip
\bcbar[label=Scan photos]{0}\\
\smallskip
\bcbar[label=Qt - try to get progress bar to move]{0}\\
\smallskip
\bcbar[label=Qt - Add combo box status for tasks]{0}\\
\smallskip
\bcbar[label=Weight]{0}\\
\smallskip
\bcbar[label=Weight]{0}\\
\smallskip
\bcbar[label=Weight]{0}\\
\smallskip
\bcbar[label=Weight]{0}\\
\smallskip
\bcbar[label=Weight]{0}\\
\smallskip
\bcbar[label=Weight]{0}\\
\smallskip
\bcbar[label=Weight]{0}\\
\smallskip
\bcbar[label=Weight]{0}\\
\smallskip
\bcbar[label=Weight]{0}\\
\smallskip
\bcbar[label=Weight]{0}\\
\smallskip
\bcbar[label=Weight]{0}\\
\smallskip
\bcbar[label=Weight]{0}\\
\smallskip
\bcbar[label=Weight]{0}\\
\smallskip
\bcbar[label=Weight]{0}\\
\smallskip
\bcbar[label=Weight]{0}\\
\smallskip
\bcbar[label=Weight]{0}\\
\smallskip
\bcbar[label=Qt - Add small calendar for start and end dates]{0}\\
\smallskip
\bcbar[label=Weight]{0}\\
\smallskip
\bcbar[label=Weight]{0}\\
\smallskip
\bcbar[label=Qt - Add combo box prority for tasks]{0}\\
\smallskip
\bcbar[label=Qt - Add small calendar for start and end dates]{0}\\
\smallskip
\bcbar[label=Scilab - Manage interpolation]{0}\\
\smallskip
\bcbar[label=Qt - add completion for tasks research]{0}\\
\smallskip
\bcbar[label=Joomla - Fill in the pages automatically]{0}\\
\smallskip
\bcbar[label=]{0}\\
\smallskip
\bcbar[label=Joomla - Implement an insert into the database]{0}\\
\smallskip
\bcbar[label=Qt update colors of tasks depending on priority]{0}\\
\smallskip
\bcbar[label=Installer NetBeans]{0}\\
\smallskip
\bcbar[label=Add the nice graph to the Stocks paper]{0}\\
\smallskip
\bcbar[label=]{0}\\
\smallskip
\bcbar[label=]{0}\\
\smallskip
\bcbar[label=]{0}\\
\smallskip
\bcbar[label=]{0}\\
\smallskip
\bcbar[label=]{0}\\
\smallskip
\bcbar[label=Weight]{0}\\
\smallskip
\bcbar[label=Weight]{0}\\
\smallskip
\bcbar[label=Weight]{0}\\
\smallskip
\bcbar[label=Clean the Startocaster]{0}\\
\smallskip
\bcbar[label=Plant seeds]{0}\\
\smallskip
\bcbar[label=Clean the Startocaster]{0}\\
\smallskip
\bcbar[label=List accesses]{0}\\
\smallskip
\bcbar[label=Plant seeds]{0}\\
\smallskip
\bcbar[label=Pay Alfa]{0}\\
\smallskip
\bcbar[label=Manage historical revenues]{0}\\
\smallskip
\bcbar[label=Transfert MDP agenda 2014]{0}\\
\smallskip
\bcbar[label=Declare revenues]{0}\\
\smallskip
\bcbar[label=Use lifting barr]{0}\\
\smallskip
\bcbar[label=Build workbench]{0}\\
\smallskip
\bcbar[label=Buy a camping chair]{0}\\
\smallskip
\bcbar[label=Add birthdays on the calendar]{0}\\
\smallskip
\bcbar[label=Sell guitar stand]{0}\\
\smallskip
\bcbar[label=Regroup contacts by town]{0}\\
\smallskip
\bcbar[label=Check for storage]{0}\\
\smallskip
\bcbar[label=Polish glasses]{0}\\
\smallskip
\bcbar[label=Add Linux & News & Sports & Admin]{0}\\
\smallskip
\bcbar[label=Add St Meen to the calendrier]{0}\\
\smallskip
\bcbar[label=Pay baillif for the Castel harbour]{0}\\
\smallskip
\bcbar[label=visiter les terrains et poser des jalons]{0}\\
\smallskip
\bcbar[label=Make winter covers for Plijadur]{0}\\
\smallskip
\bcbar[label=Change the strings]{0}\\
\smallskip
\bcbar[label=Prices index]{0}\\
\smallskip
\bcbar[label=Complete description]{0}\\
\smallskip
\bcbar[label=Link to the vendor - http://www.jeanneau.fr/]{0}\\
\smallskip
\bcbar[label=Percentage Broker]{0}\\
\smallskip
\bcbar[label=Clean up external company]{0}\\
\smallskip
\bcbar[label=Price 30]{0}\\
\smallskip
\bcbar[label=For sale Jeanneau]{0}\\
\smallskip
\bcbar[label=Photo Facebook + Photos Blog + Videos]{0}\\
\smallskip
\bcbar[label=Check the size of the pictures and homogenize]{0}\\
\smallskip
\bcbar[label=Prices index]{0}\\
\smallskip
\bcbar[label=Interior clean up]{0}\\
\smallskip
\bcbar[label=Update LinkedIn]{0}\\
\smallskip
\bcbar[label=Get back stuff Jessica]{0}\\
\smallskip
\bcbar[label=scanners les docs]{0}\\
\smallskip
\bcbar[label=Photos of my stuff]{0}\\
\smallskip
\bcbar[label=Add consumption to the IT tools]{0}\\
\smallskip
\bcbar[label=Make list of books]{0}\\
\smallskip
\bcbar[label=Write log book]{0}\\
\smallskip
\bcbar[label=Find back crew Plijadur]{0}\\
\smallskip
\bcbar[label=Ben Harper tatoo]{0}\\
\smallskip
\bcbar[label=Create a football team]{0}\\
\smallskip
\bcbar[label=Contact PHD]{0}\\
\smallskip
\bcbar[label=Call Christine]{0}\\
\smallskip
\bcbar[label=Sell guitar]{0}\\
\smallskip
\bcbar[label=Calculate Australian project]{0}\\
\smallskip
\bcbar[label=Meeting taxes]{0}\\
\smallskip
\bcbar[label=Get a lawyer]{0}\\
\smallskip
\bcbar[label=Get the musculation stuff up stairs]{0}\\
\smallskip
\bcbar[label=Buy guitar Titi]{0}\\
\smallskip
\bcbar[label=Sale genoa]{0}\\
\smallskip
\bcbar[label=Check keel]{0}\\
\smallskip
\bcbar[label=Rent of the boat]{0}\\
\smallskip
\bcbar[label=Check VHF]{0}\\
\smallskip
\bcbar[label=Check nautical maps]{0}\\
\smallskip
\bcbar[label=Sale genoa Lorient]{0}\\
\smallskip
\bcbar[label=Repair electric rack]{0}\\
\smallskip
\bcbar[label=Faire reparer la clim de l'alfa]{0}\\
\smallskip
\bcbar[label=Clean up the car]{0}\\
\smallskip
\bcbar[label=Buy black gloves]{0}\\
\smallskip
\bcbar[label=Installer musculation engine]{0}\\
\smallskip
\bcbar[label=Get back the portable freeezer]{0}\\
\smallskip
\bcbar[label=Get back tools from Zaz]{0}\\
\smallskip
\bcbar[label=Acheter cadre]{0}\\
\smallskip
\bcbar[label=Change oil]{0}\\
\smallskip



%\subsubsection{ROI}
\subsubsection{Sponsors}
%\includegraphics[width=90mm]{Contacts.png}
%\includegraphics[width=90mm]{Projects.png}
%\includegraphics[width=90mm]{EventsPro.png}
%\includegraphics[width=90mm]{Resources.png}
%\includegraphics[width=90mm]{Resources.png}
\input{kiviat5}
%\subsubsection{Budget}
%From Tasks.csv
\subsubsection{Risks}
Not enough budget, Not enough resources (including me)
\subsubsection{Targets}
From Gantt
\subsubsection{Propositions from Artificial Intelligence}
From Gantt
%\subsubsection{Table}


\section{Associated Project Management}

From now on we focus on one single project, else it's gonna be messie 
\subsection{Deliverables}
From events.csv
\subsection{Gantt}
Pour que le Gantt soit a sa place!!!!!
%\input{Gantt2}
%#####################################################################################

\subsection{Burn down}
From tasks.csv + Dispo weekly of each resource
%Graphique\\
\\
Calcul de vitesse\\
s=speed\\
p=position\\
n=number of units\\
t=time\\
s=n/t\\
\\
Projection lineaire\\
y=s*x+p\\
Theoretical end\\

\pgfmathsetseed{1138} % set the random seed
\pgfplotstableset{ % Define the equations for x and y
    create on use/x/.style={create col/expr={42+2*\pgfplotstablerow}},
    create on use/y/.style={create col/expr={(0.6*\thisrow{x}+130)+5*rand}}
}
% create a new table with 30 rows and columns x and y:
\pgfplotstablenew[columns={x,y}]{30}\loadedtable
\begin{tikzpicture}[scale=0.9]
  \begin{axis} [
      xlabel     = Weight (kg), % label x axis
      ylabel     = Height (cm), % label y axis
      axis lines = left, %set the position of the axes
      clip       = false, 
      xmin = 40,  xmax = 105, % set the min and max values of the x-axis
      ymin = 150, ymax = 200, % set the min and max values of the y-axis
    ]
    \addplot [only marks] table {\loadedtable};
    \addplot [no markers, thick, red]
      table [y={create col/linear regression={y=y}}] {\loadedtable}
      node [anchor=west] {$\pgfmathprintnumber[precision=2, fixed zerofill]
      {\pgfplotstableregressiona} \cdot \mathrm{Weight} +
      \pgfmathprintnumber[precision=1]{\pgfplotstableregressionb}$};
  \end{axis}
\end{tikzpicture}


%\includegraphics[width=20pts]{Geometric.jpg}

\subsubsection{Perfect Burndown}
%\input{2Din3D}
\input{2Din3D_new}

%\subsubsection{Yet another Perfect Burndown}
%\input{Scatter2}

\subsection{Resources}

Top 10 resources to take care of\\
Workload (days, percentage time, stress, fatigue sum of days without break, ambition)
Diagramme en etoile per deliverable\\
Rates for all the resources

\subsubsection{Data}
\begin{longtable}{|c|c|c|c|c|}
\hline
\multicolumn{5}{|c|}{Contacts} \\
\hline
ID & Name & Rating & Town & Telephone\\
\hline
104 & Wing & 1000 & Paris & 85296001395\\
\hline
112 & Zazzy & 900 & World & 33611037735\\
\hline
29 & Flozio & 850 & London & 33680938975\\
\hline
25 & Fanch & 800 & Brest & 33685769214\\
\hline
70 & Mummy & 800 & Paris & 33674340108\\
\hline
92 & Sonia & 600 & Paris & 33612425946\\
\hline
86 & Rico & 550 & Paris & 33611037992\\
\hline
62 & Maeva & 500 & Paris & 33662575512\\
\hline
44 & Jessie & 500 & Paris & 33681775660\\
\hline
43 & Jessica & 500 & Paris & 699114929\\
\hline
24 & Etienne & 400 & Brest & 619144832\\
\hline
2 & Armelle & 400 & HongKong & 33677134026\\
\hline
99 & Tiprig & 250 & Paris & 33663626188\\
\hline
23 & Elise & 250 & Brest & 33614921085\\
\hline
 ... & ... & ... & ... & ... \\
\hline
Total & 8300 &  & & \\
\hline
\end{longtable}

%\subsubsection{Graph}
%\begin{bchart}[min=0,max=1000,step=200,unit=K\texteuro]
\bcbar[label=Pascalapo]{1000}\\
\smallskip
\bcbar[label=Fredport]{950}\\
\smallskip
\bcbar[label=Patrickport]{600}\\
\smallskip
\bcbar[label=Cathyapo]{500}\\
\smallskip
\bcbar[label=Neilport]{300}\\
\smallskip
\bcbar[label=C�dricGarApo]{250}\\
\smallskip
\bcbar[label=Arnaudport]{200}\\
\smallskip
\bcbar[label=Angeport]{180}\\
\smallskip
\bcbar[label=C�dricLApo]{100}\\
\smallskip
\bcbar[label=Nicoport]{50}\\
\smallskip
\bcbar[label=SebTapo]{50}\\
\smallskip
\end{bchart}

\subsubsection{Cheese}
\begin{tikzpicture}[scale=2.5]
\foreach \p/\t in {
23 / Pascalapo-1000K\texteuro ,
22 / Fredport-950K\texteuro ,
14 / Patrickport-600K\texteuro ,
11 / Cathyapo-500K\texteuro ,
7 / Neilport-300K\texteuro ,
5 / C�dricGarApo-250K\texteuro ,
4 / Arnaudport-200K\texteuro ,
4 / Angeport-180K\texteuro ,
2 / C�dricLApo-100K\texteuro ,
1 / Nicoport-50K\texteuro ,
1 / SebTapo-50K\texteuro ,
}
  {
\setcounter{a}{\value{b}}
\addtocounter{b}{\p}
\slice{\thea/100*360}
          {\theb/100*360}
          {\p\%}{\t}
  }
\end{tikzpicture}


\subsubsection{Kiviat}
\begin{tikzpicture}[scale=0.5]
\tkzKiviatDiagramFromFile[
        scale=.5,
        label distance=.5cm,
        gap     = 1,
        label space=3,  
        lattice = 10]{resourcesKiviat.dat}
\tkzKiviatLineFromFile[thick,
                       color      = blue,
                       mark       = ball,
                       ball color = blue,
                       mark size  = 4pt,
                       fill       = blue!20]{resourcesKiviat.dat}{2}
\tkzKiviatLineFromFile[thick,
                       color      = red,
                       mark       = ball,
                       ball color = red,
                       mark size  = 4pt,
                       fill       = red!20]{resourcesKiviat.dat}{1}     
\end{tikzpicture}


\subsubsection{Star diagram}
%\setlength{\unitlength}{0.75mm}
%\begin{picture}(80,60)
%\put(30,20){\vector(1,0){30}}
%\put(30,20){\vector(4,1){20}}
%\put(30,20){\vector(3,1){25}}
%\put(30,20){\vector(2,1){30}}
%\put(30,20){\vector(1,2){10}}
%\put(0,3.35){\makebox(0,0){$y$}
%\thicklines
%\put(20,15){\vector(-4,1){35}}
%\put(20,15){\vector(-1,4){10}}
%\thinlines
%\put(30,20){\vector(-1,-1){5}}
%\put(30,20){\vector(-1,-4){5}}
%\end{picture}

\subsection{Tasks}

{\footnotesize
\begin{longtable}{|c|c|c|c|c|c|}
\hline
\multicolumn{6}{|c|}{Tasks} \\
\hline
Project & Task & Return & Cost & R/C & NumDays \\
\hline
Finance & Vendre Plijadur & 27000 & 1 & 27000 & 10\\
\hline
Finance & Vendre voiture & 8000 & 1 & 8000 & 10\\
\hline
Friends & Bouffe Armelle & 3200 & 5 & 640 & 10\\
\hline
Friends & Bouffe Christophe & 1200 & 2 & 600 & 10\\
\hline
Work & Rencontrer mecs tyfab & 500 & 1 & 500 & 10\\
\hline
Friends & Bouffe Etienne & 4000 & 10 & 400 & 10\\
\hline
Friends & Bouffe Zaz & 3800 & 10 & 380 & 10\\
\hline
Friends & Bouffe Jab & 400 & 2 & 200 & 10\\
\hline
Boat & revision moteur & 100 & 1 & 100 & 10\\
\hline
Boat & Carenner le Boat & 90 & 1 & 90 & 10\\
\hline
Finance & Faire le virement HSBC & 80 & 1 & 80 & 10\\
\hline
Boat & genois - crowd funding? & 50 & 1 & 50 & 10\\
\hline
\end{longtable}

}

\section{Annexes}

%\subsection{Rate}

%\begin{bchart}[min=0,max=10000,step=5000,unit=\texteuro]
%  \bcbar[label=Daily]{400}
%  \smallskip
%  \bcbar[label=Weekly]{1900}
%  \smallskip
%  \bcbar[label=Monthly]{7600}
%  \smallskip
%  \bcbar[label=3 Months]{10000}
%  \smallskip
  %\medskip
  %\bigskip
%\end{bchart}

\subsection{Tools}

{\footnotesize
\subsubsection{Data}
\begin{longtable}{|c|c|c|c|}
\hline
\multicolumn{4}{|c|}{Tools} \\
\hline
Tool & Rating & Experience & Link\\
\hline
Apache & 14 & 14 & \\
\hline
Php & 13 & 13 & \\
\hline
Perl & 12 & 12 & \\
\hline
Java & 11 & 11 & \\
\hline
Vi & 10 & 10 & \\
\hline
Jira & 9 & 9 & \\
\hline
Latex & 8 & 8 & \\
\hline
MySql & 7 & 7 & \\
\hline
Linux & 6 & 6 & \\
\hline
Scilab & 5 & 5 & \\
\hline
Clips & 4 & 4 & \\
\hline
SVN & 3 & 3 & \\
\hline
Hudson & 2 & 2 & \\
\hline
Gimp & 1 & 1 & \\
\hline
 ... & ... & ... & ... \\
\hline
Total & 105 & 105 & \\
\hline
\end{longtable}

%\subsubsection{Graph}
%\begin{bchart}[min=0,max=14,step=0.466666666666667,unit=K\texteuro]
\bcbar[label=Apache]{14}\\
\smallskip
\bcbar[label=Php]{13}\\
\smallskip
\bcbar[label=Perl]{12}\\
\smallskip
\bcbar[label=Java]{11}\\
\smallskip
\bcbar[label=Vi]{10}\\
\smallskip
\bcbar[label=Jira]{9}\\
\smallskip
\bcbar[label=Latex]{8}\\
\smallskip
\bcbar[label=MySql]{7}\\
\smallskip
\bcbar[label=Linux]{6}\\
\smallskip
\bcbar[label=Scilab]{5}\\
\smallskip
\bcbar[label=Clips]{4}\\
\smallskip
\bcbar[label=SVN]{3}\\
\smallskip
\bcbar[label=Hudson]{2}\\
\smallskip
\bcbar[label=Gimp]{1}\\
\smallskip
\end{bchart}

\subsubsection{Cheese}
\begin{tikzpicture}[scale=3]
\foreach \p/\t in {
13 / Apache-14K\texteuro ,
12 / Php-13K\texteuro ,
11 / Perl-12K\texteuro ,
10 / Java-11K\texteuro ,
9 / Vi-10K\texteuro ,
8 / Jira-9K\texteuro ,
7 / Latex-8K\texteuro ,
6 / MySql-7K\texteuro ,
5 / Linux-6K\texteuro ,
4 / Scilab-5K\texteuro ,
3 / Clips-4K\texteuro ,
2 / SVN-3K\texteuro ,
1 / Hudson-2K\texteuro ,
0 / Gimp-1K\texteuro ,
}
  {
\setcounter{a}{\value{b}}
\addtocounter{b}{\p}
\slice{\thea/100*360}
          {\theb/100*360}
          {\p\%}{\t}
  }
\end{tikzpicture}

\subsubsection{Architecture}
}

\subsection{Documentations and links}
See References-ok.csv
%\section{Calendars}

%\subsection{Monthly calendar}

%{\footnotesize
%Will contain deliverables, meetings and holidays, republic off\\
%}

%\begin{calendar}{\hsize}
 
%----------------------------------------------------------------------------------------
%	BLANK DAYS BEFORE THE BEGINNING OF THE CALENDAR
%----------------------------------------------------------------------------------------

% This part is very finicky. It defines the number of blank days at the beginning of the calendar before the first of the month starts. If you need this to be more than 4 (i.e. the first starts on a Friday or Saturday in a 31 day month), then you have two options: 
% 1) You can uncomment another one or two \BlankDay's below which will make a new week (6 total) which makes the calendar too big for one page, remedy this by decreasing the size of each day by replacing 2.5cm below with a smaller number. 
% 2) Make the spill-over days start at the top left of the calendar (i.e. the calendar starts with 31 then a few days blank then 1, 2, 3, etc). The second option can be configured by uncommenting the below:

%\setcounter{calendardate}{31} % Begin the count with 31 so the top left day is 31; this can be changed to 29 or 30 as required
%\day{}{\vspace{2.5cm}} % 31 - add another line identical to this if starting at 30 or earlier

% You will need to comment out the 31 in the NUMBERED DAYS AND CALENDAR CONTENT section below for this as well as commenting out one of the \BlankDay's below. Play around with it and you will get it.

%\BlankDay
%\BlankDay
%\BlankDay
%\BlankDay
%\BlankDay
%\BlankDay

%----------------------------------------------------------------------------------------
%	NUMBERED DAYS AND CALENDAR CONTENT
%----------------------------------------------------------------------------------------

% These are the numbered days in the template - if there are less than 31 days simply comment out the bottom lines.

% \vspace{2.5cm} is only there to provide an even look to the calendar where each day is 2.5cm tall, it can be changed or removed to automatically adjust to the day in the week with the most content

%\setcounter{calendardate}{1} % Start the date counter at 1

%\day{Work}{10am Meeting with Boss \\[6pt] 12pm Meeting with Group} % 1 - Example of content
%\day{}{\vspace{2.5cm}} % 2 
%\day{}{\vspace{2.5cm}} % 3
%\day{}{\vspace{2.5cm}} % 4
%\day{}{\vspace{2.5cm}} % 5
%\day{}{\vspace{2.5cm}} % 6
%\day{}{\vspace{2.5cm}} % 7
%\day{}{\vspace{2.5cm}} % 8
%\day{Work}{Start giving classes \\[6pt] Pythagore} % 1 - Example of content
%\day{}{\vspace{2.5cm}} % 9
%\day{Work}{Delivery one \\[6pt] 12pm Meeting with Group} % 1 - Example of content
%\day{}{\vspace{2.5cm}} % 10
%\day{}{\vspace{2.5cm}} % 11
%\day{}{\vspace{2.5cm}} % 12
%\day{Work}{Delivery two \\[6pt] 12pm Meeting with Group} % 1 - Example of content
%\day{}{\vspace{2.5cm}} % 13
%\day{}{\vspace{2.5cm}} % 14
%\day{Perso}{Annif \\[6pt] Cado} % 1 - Example of content
%\day{}{\vspace{2.5cm}} % 15
%\day{}{\vspace{2.5cm}} % 16
%\day{}{\vspace{2.5cm}} % 17
%\day{}{\vspace{2.5cm}} % 18
%\day{}{\vspace{2.5cm}} % 19
%\day{}{\vspace{2.5cm}} % 20 
%\day{}{\vspace{2.5cm}} % 21
%\day{Perso}{Annif \\[6pt] Cado} % 1 - Example of content
%\day{}{\vspace{2.5cm}} % 22
%\day{}{\vspace{2.5cm}} % 23
%\day{}{\vspace{2.5cm}} % 24
%\day{}{\vspace{2.5cm}} % 25
%\day{}{\vspace{2.5cm}} % 26
%\day{}{\vspace{2.5cm}} % 27
%\day{}{\vspace{2.5cm}} % 28
%\day{}{\vspace{2.5cm}} % 29 
%\day{}{\vspace{2.5cm}} % 30 
%\day{}{\vspace{2.5cm}} % 31

% Un-comment the \BlankDay below if the bottom line of the calendar is missing
%\BlankDay

% Un-comment to start counting again after 31
%\setcounter{calendardate}{1}
%\day{}{\vspace{2.5cm}} % 1
%\day{}{\vspace{2.5cm}} % 2
%\day{}{\vspace{2.5cm}} % 3

%----------------------------------------------------------------------------------------

%\finishCalendar
%\end{calendar}

%\subsection{Yearly calendar}

%Mon calendrier annuel\\

\end{document}
