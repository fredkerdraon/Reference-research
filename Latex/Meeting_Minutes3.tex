\documentclass[a4paper, 11pt]{article}
\usepackage{comment} % enables the use of multi-line comments (\ifx \fi) 
\usepackage{fullpage} % changes the margin
\usepackage[swedish]{babel}
\usepackage[utf8]{inputenc}
\usepackage{graphicx}
\usepackage{multicol}
\usepackage{float}
\usepackage{fancyhdr}
\usepackage{enumitem}
\pagestyle{fancy} 
\usepackage{pdfpages}
%\usepackage[head=128pt]{geometry}
\title{Stormötesprotokollmall ES}
\author{Erik Jonasson}
\usepackage{geometry}
\setlength{\footskip}{0.1pt}
\setlength{\headheight}{80pt}
\setlength{\topmargin}{0pt}
\setlength\parindent{0pt}
\fancypagestyle{style1}{
\lhead{\includegraphics[width=7cm]{ES_Logo_V2.png}}
\rhead{Stormötesprotokoll 20xx-xx-xx\\
Föreningen Energisystemteknologerna\\
Uthgård, Polacksbacken hus 73, 75237 Uppsala\\}
\renewcommand{\headrulewidth}{0pt}
\cfoot{
\makebox{}\\
\makebox{}\\
\makebox[0.1\linewidth]{\rule{0.1\linewidth}{0.1pt}} \hspace{1cm} \makebox[0.1\linewidth]{\rule{0.1\linewidth}{0.1pt}} \hspace{1cm} \makebox[0.1\linewidth]{\rule{0.1\linewidth}{0.1pt}} \hspace{1cm} \makebox[0.1\linewidth]{\rule{0.1\linewidth}{0.1pt}} \hspace{1cm}\\}
}
\fancypagestyle{style2}{
\lhead{\includegraphics[width=7cm]{ES_Logo_V2.png}}
\rhead{Stormötesprotokoll 20xx-xx-xx\\
Föreningen Energisystemteknologerna\\
Uthgård, Polacksbacken hus 73, 75237 Uppsala\\}
\renewcommand{\headrulewidth}{0pt}
\cfoot{}
}
\begin{document}
\pagestyle{style1}
% \lhead{\includegraphics[width=7cm]{ES_Logo_V2.png}}
% \rhead{Stormötesprotokoll 20xx-xx-xx\\
% Föreningen Energisystemteknologerna\\
% Uthgård, Polacksbacken hus 73, 75237 Uppsala\\}
% \cfoot{
% \makebox{}\\
% \makebox{}\\
% \makebox[0.1\linewidth]{\rule{0.1\linewidth}{0.1pt}} \hspace{1cm} \makebox[0.1\linewidth]{\rule{0.1\linewidth}{0.1pt}} \hspace{1cm} \makebox[0.1\linewidth]{\rule{0.1\linewidth}{0.1pt}} \hspace{1cm} \makebox[0.1\linewidth]{\rule{0.1\linewidth}{0.1pt}} \hspace{1cm}\\}

\textbf{Datum:} 20xx-xx-xx\\
\textbf{Tid:} xx.xx-xx.xx\\
\textbf{Plats:} Sal, Område\\

\textbf{Bilagor:}\\ Bilaga 1 - Dagordning\\ %Skall alltid bifogas som bilaga
Bilaga 2 - Motioner, revisionberättelser etc\\ %Se stadgarna för aktuella bilagor

\makebox[\linewidth]{\rule{\linewidth}{0.4pt}}\\
\textbf{Närvarande:}\hfill 39 medlemmar i FET\\
\makebox[\linewidth]{\rule{\linewidth}{0.4pt}}\\

\section*{§1. Mötets öppnande}
Mötesordförande förklarar mötet öppnat kl xx.xx.

\section*{§2. Godkännande av röstlängd}
Stormötet beslutar\\
\textbf{\textit{att}} fastställa röstlängden till xx röstberättigade.\\

\section*{§3. Exempel}
Stormötet beslutar\\
 \textbf{\textit{att}}

\section*{§4. Exempel}
Stormötet beslutar\\
 \textbf{\textit{att}}

\section*{§5. Exempel på punkt med yrkande}
Styrelsen yrkar\\
\textbf{\textit{att}} lägga till punkten ``Föregående stormötesprotokoll'' som §7.\\

\noindent Stormötet beslutar\\
\textbf{\textit{att}} justera dagordningen enligt styrelsens yrkanden.\\
\textbf{\textit{att}} godkänna dagordningen.

\section*{§6. Exempel med punktlista, val av stormötesfunktionärer} 
\begin{enumerate}[label=(\alph*)]
    \item Stormötesordförande\\
    
    Stormötet beslutar\\
    \textbf{\textit{att}} välja Person Personson till mötesordförande för stormötet våren 2017. 
    \item Stormötessekreterare\\
    
    Stormötet beslutar\\
    \textbf{\textit{att}} välja Namn Namnson till mötessekreterare för stormötet våren 2017.
    \end{enumerate}
 \section*{§7. Viktiga tips för protokollet}
 \begin{enumerate}[label=(\alph*)]
 \item Skriv gärna anteckningar i ett annat dokument än detta och gör det sedan snyggt
 \item Om yrkanden kommer in under mötets gång, se till att de finns i skrift och noteras i protokollet
 \item Notera alltid klockslag om röstlängden justeras. 
 \item Notera alltid klockslag då en punkt återupptas, om aktuellt
 \item Ovanstående är för att det alltid ska gå att följa i protokollet vad röstlängden är för ett givet beslut, då detta regleras (iofs svagt men ändå) i stadgarna
    \end{enumerate}
    
    \section*{§8. Viktiga tips för användandet av detta dokument}
    \begin{enumerate}[label=(\alph*)]
    \item Även om du inte är helt bekväm med \LaTeX försök gärna använd denna mall så att protokollen genom åren blir enhetliga.
    \item Det mesta borde vara självförklarande, men att ändra headern och footern kan vara lite knepigt, så var förstig med ändringar där
    \item De fyra strecken på varje sida är för att mötesordförande, sekretare och justeringsmän ska signera med sina initialer på varje sida
    \item För att ta bort initalsigneringsstecken på sista sidan där fullständig signering sker, skriv 	\verb+\thispagestyle{style2}+ någonstans på den sidan. Det är formaterat så att style1 är med både header och footer, style2 är endast med header.
    \end{enumerate}
\newpage
    \section*{§23. Mötets avslutande}
Elin Luedtke avslutar mötet kl 16.34..
\thispagestyle{style2}
\makebox{}\\
\makebox{}\\
\makebox{}\\
\makebox[0.4\linewidth]{\rule{0.4\linewidth}{0.4pt}} \hspace{1cm} \makebox[0.4\linewidth]{\rule{0.4\linewidth}{0.4pt}} \hspace{1cm}\\
\makebox[0.4\linewidth]{Sek-namn} \hspace{1cm}
\makebox[0.4\linewidth]{Ordf-namn} \hspace{1cm}\\
\makebox[0.4\linewidth]{Mötessekreterare} \hspace{1cm}
\makebox[0.4\linewidth]{Mötesordförande} \hspace{1cm}\\
\makebox{}\\
\makebox{}\\
\makebox{}\\
\noindent\makebox[0.4\linewidth]{\rule{0.4\linewidth}{0.4pt}} \hspace{1cm} \makebox[0.4\linewidth]{\rule{0.4\linewidth}{0.4pt}} \hspace{1cm}\\
\makebox[0.4\linewidth]{Justerare 1} \hspace{1cm}
\makebox[0.4\linewidth]{Justerare 2} \hspace{1cm}\\
\makebox[0.4\linewidth]{Justerare} \hspace{1cm}
\makebox[0.4\linewidth]{Justerare} \hspace{1cm}\\



\end{document}
