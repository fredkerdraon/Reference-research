% !TEX TS-program = pdflatex
% !TEX encoding = UTF-8 Unicode

% This is a simple template for a LaTeX document using the "article" class.
% See "book", "report", "letter" for other types of document.

\documentclass[11pt]{article} % use larger type; default would be 10pt

\usepackage[utf8]{inputenc} % set input encoding (not needed with XeLaTeX)
\usepackage{pgfplots}
\usepackage{tikz}
\usepackage{dtklogos}
\usepackage{tikz}
\usetikzlibrary{mindmap,shadows}
\usepackage[hidelinks,pdfencoding=auto]{hyperref}

%%% Examples of Article customizations
% These packages are optional, depending whether you want the features they provide.
% See the LaTeX Companion or other references for full information.

%%% PAGE DIMENSIONS
\usepackage{geometry} % to change the page dimensions
\geometry{a4paper} % or letterpaper (US) or a5paper or....
% \geometry{margin=2in} % for example, change the margins to 2 inches all round
% \geometry{landscape} % set up the page for landscape
%   read geometry.pdf for detailed page layout information

\usepackage{graphicx} % support the \includegraphics command and options

% \usepackage[parfill]{parskip} % Activate to begin paragraphs with an empty line rather than an indent

%%% PACKAGES
\usepackage{booktabs} % for much better looking tables
\usepackage{array} % for better arrays (eg matrices) in maths
\usepackage{paralist} % very flexible & customisable lists (eg. enumerate/itemize, etc.)
\usepackage{verbatim} % adds environment for commenting out blocks of text & for better verbatim
\usepackage{subfig} % make it possible to include more than one captioned figure/table in a single float
% These packages are all incorporated in the memoir class to one degree or another...

%%% HEADERS & FOOTERS
\usepackage{fancyhdr} % This should be set AFTER setting up the page geometry
\pagestyle{fancy} % options: empty , plain , fancy
\renewcommand{\headrulewidth}{0pt} % customise the layout...
\lhead{}\chead{}\rhead{}
\lfoot{}\cfoot{\thepage}\rfoot{}

%%% SECTION TITLE APPEARANCE
\usepackage{sectsty}
\allsectionsfont{\sffamily\mdseries\upshape} % (See the fntguide.pdf for font help)
% (This matches ConTeXt defaults)

%%% ToC (table of contents) APPEARANCE
\usepackage[nottoc,notlof,notlot]{tocbibind} % Put the bibliography in the ToC
\usepackage[titles,subfigure]{tocloft} % Alter the style of the Table of Contents
\renewcommand{\cftsecfont}{\rmfamily\mdseries\upshape}
\renewcommand{\cftsecpagefont}{\rmfamily\mdseries\upshape} % No bold!

%%% END Article customizations

%%% The "real" document content comes below...

\title{Reflexions}
\author{Frederic Kerdraon}
%\date{} % Activate to display a given date or no date (if empty),
         % otherwise the current date is printed 

\begin{document}
\maketitle

\section{Introduction}

Cela fait déjà plusieurs années que j'éprouve le besoin d'écrire mes réflexions sur le monde qui m'entoure.
Pour que cette manifestation puisse s'opérer (la première fois en tout cas), des conditions particulières doivent être à l'oeuvre, et jusqu'ici les étoiles ne semblaient pas décidées à s'aligner.
Les circonstances actuelles font que le temps m'est permis et que mes méditations récentes, fruit d'années et d'années de tergiversations intellectuelles de toutes sortes ont fait que je me suis lancé dans la réalisation de plusieurs livrables aboutis, somme d'un travail hallucinant de reflexions, de mise à l'épreuve, et de conclusions personelles.
Il est temps maintenant de se lancer...
\subsection{Objectifs}
En décalage avec la société actuelle, je n'y retrouve plus les valeurs les plus basiques nécessaires à la cohabitation naturelle d'être humains:\\\\
- \textbf{La liberté}\\
- \textbf{L'égalité}\\
- \textbf{La fraternité}\\

Les petits fils de putes Bretons sont tous morts.
Ils te font croire que tu fais partie de la famille, qu'on est ensemble, mais ils te lachent dès que ça entame un tant soit peu leur petits intérets.
Des cochons capitalistes!
Il faudrait maintenant assumer que la nature est cruelle, les religions n'ont été inventées que pour les dociles, et Bruce Lee, Mike Tyson et Alexandre le Grand sont mes idoles.
Didier Sicre, Catherine Le Flohic et Laurent Thiebaut doivent payer pour apprendre. Lionel aussi d'ailleurs.
Les affaires de pédophilie nous en apprenne beaucoup sur la "morale" et la littérature.
Action directe, Brigitte Lahaie, Mesrine et L'ultra gauche contre Castaner.
Albert Camus, Karl Marx, Frida Khalo, Leon Trotsky, Diego Rivera, Mere Theresa, Andreei Sakharov.
Trump on l'encule!
Le camp de concenntration capitaliste est bien vérouillé.
Il n'est plus possible de s'adresser aux femmes quand on est un homme, elles ont peur.
"Il est temps aussi de mettre aux oubliettes les haines viscérales des femmes envers les hommes." Tssssss... mais de quoi parlez-vous?... c'est d'une telle tristesse de croire que les voix qui se libèrent aujourd'hui sont une croisade intestine contre les hommes alors qu'elles ne visent qu'à mettre au jour et stopper les sales types qui par leur comportement, - et lui seul - créent des victimes bien réelles chez les femmes et portent un tord quasi irréparable aux hommes qui se conduisent comme des hommes.
Irréparable parce que des femmes et des hommes s'indignent quand les sales types sont mis au pied du mur de leur abjection, foulant de leurs pieds brutaux le coeur des femmes meurtries.
Quel dommage d'aimer si peu ses consoeurs, si peu les hommes de bien.

Les libertés individuelles sont maintenant très très compromises par une organisation sociétale qui définit complètement nos vies, et légifère sur le moindre détail (sans conditions!). Le dogme est devenu coercitif à l'extrême et il devient presque impossible d'ETRE pleinement, sauf à avoir une fortune personelle, nous mettant à l'abri des besoins d'adhérer aux codes de la société telle qu'elle s'est définie au cours des années.
Le "progrès" a enfermé les gens dans des besoins de plus en plus importants, et de plus en plus aliénants, à grand coup de lavage de cerveau via les médias d'information.
La quête matérialiste a un sens lorsqu'elle est partagée de façon égalitaire, et pour assouvir des besoins réels, non pas pour se distinguer en tant qu'individu, ou pire, pour exister en tant qu'individu.
Nous utilisons certains "articles" que nous possédons avec une fréquence dérisoire, mais étant donné que ceux là nous sont réellement nécessaires à cette occasion, il nous semble important d'en disposer.
Il m'a toujours semblé que de disposer d'une machine à laver, d'un sèche linge et de tous les autres articles de ce genre est absolument indispensable si l'on veut rester propre, bien habillé, au chaud et bien nourri. Toutefois cela ne peut largement être partagé au sein d'un groupe, et l'économie très substancielle qui s'accompagnerait de cette démarche, nous permettrait d'affecter les ressources préservées à d'autres objectifs afin de nous développer. 
Pourquoi n'est-il en apparence pas possible de le faire?
Pourquoi le mode de vie qui est le notre fait-il que nous en soyons arrivé à ce stade?
La lecture du manifeste du parti communiste de Karl Marx (sans entrer ici dans l'application d'un dogme, dogme que je hais de quelque provenance qu'il vienne) ne m'a paru qu'une série de propositions complètement logiques et évidentes, toutefois on nous bassine depuis toujours avec Staline, l'association des communistes avec Satan, les communautés avec des sectes, l'autonomie avec la folie, etc. Les éléments de langage sont à ce titre très informatifs. Francois Hollande lorsqu'il décrivait les autonomistes Catalans parlait "d'extrémistes", comme tous ceux qui veulent sortir de l'Union Européenne par exemple sont allègrement dénigré et travestis en fascites, anarchistes, et j'en passe des meilleures, ils sont marginalisés et ridiculisés par "l'Establishment". Les très rares documentaires sur l'argent filtré par l'Union Européenne, et les montagnes d'Euro qu'elle distribue aux hommes politiques (sans compter l'argent de Loby) sont extrèmement révélateurs. Les députés ont l'assurance d'une position confortable (pas besoin d'être élu, pas besoin de rendre des comptes) et ne viennent pointer que pour toucher leurs jetons, sans même assiter aux débats. 
Tous les signes extérieurs que nous recevons vont dans ce sens. Que ce soit par le biais de la télévision (sic), les articles journalistiques, et maintenant via internet.
Il m'est apparu de plus en plus clair que très peu de gens sont capables de se détacher de cette propagande, et de penser librement, par eux-même, et pour eux-même, et même cette très faible proportion des gens est devenue incapable de s'extirper de ce modèle
Afin de comprendre cette tendance de l'histoire, et éviter de sombrer dans la szchizophrénie la plus totale, il convient de reprendre à la base les nécessités les plus élémentaires de l'homme, et de faire le tri avec ce qui nous est imposé par une minorité de psychopathes obnubilés par les milliards engloutis sur leurs comptes en banque, sans aucun remords ni conscience des conséquences sur les autres être humains, la nature, et plus généralement la vie sur terre.
Eux ont réussi à assurer leurs propres besoins pour des millénaires, mais sont-ils conscients qu'ils ne sont pas éternels? Pourquoi cette recherche d'avoir toujours plus alors qu'on a déjà bien plus que ce dont aurait besoin? L'orgueil? La peur qu'on leur prenne tout s'ils acceptaient d'en partager une partie?
Parmi les vices identifiés par l'espèce humaine au cours de siècles, je n'en vois pas d'autres succeptible d'expliquer la situation.\\

L'injustice gouverne ce monde, et la survie est la seule loi qui règne.

On m'a ordonné d'être docile, mais la loi du plus fort est toujours la meilleure.\\
\subsection{Les besoins fondamentaux}
L'homme a quelques besoins très bien identifiés pour "survivre" (dans l'ordre):\\\\
- L'air qu'il respire\\
- L'eau qu'il boit\\
- La nourriture qu'il mange\\
- Un abri pour se protéger des éléments\\

\footnote{De celui qui dans la bataille a vaincu mille milliers d'hommes et de celui qui s'est vaincu lui-même, c'est ce dernier qui est le plus grand vainqueur. Bouddha}

S'il réussit à pourvoir à ces éléments essentiels, il peut envisager de "survivre" quelques temps, voir même de longues années s'il s'assure de la qualité de ceux-ci avec le plus grand soin.
Toutefois celle-ci est maintenant bien souvent compromise par l'activité dite "Industrielle", et en particulier depuis la fin du 19eme siècle, et la révolution du même nom.
Inutile de faire la liste des éléments qui ont produit cette situation, elle est au coeur de l'actualité à ce moment, sans que "la gouvernance" ne décide de changer de cap (comme le capitaine d'un bateau face à une vague de travers). Encore une fois les intérêts matériels à court terme prennent le pas sur le reste, et les lobies du pétrole, de la chimie et de la pharmacie par exemple prennent en otages les pays.
Si je reprends l'analogie avec la conduite d'un navire (et je m'y connais un tout petite peu), il s'agit dans ces moments, d'anticiper avant d'en arriver là, de communiquer la situation à l'équipage, et d'avoir tout le monde sur le pont afin que la manoeuvre se passe dans les meilleures conditions.
Toutefois dans le cas de la préservation des éléments nécessaires à la vie sur terre, la solution consiste à se passer du "confort" matériel, ie une série d'objets nécessaires, mais que l'on pourrait partager facilement), et toute une liste d'autres objets matériels ou immatériels que l'on appelle loisirs. 

\subsubsection{L'air}
Constat: L'air que l'on respire est de plus en plus toxique pour la présence de la vie sur terre, qu'elle soit d'origine humaine, animale ou même pour les plantes.
Pourtant ce besoin est le plus fondamental! Ne dit-on pas qu'on est mort lorsque l'on ne respire plus? Cela n'est pas vrai pour tous les autres besoins fondamentaux, ce constat devrait donc alarmer tous les être humains sans exception, et faire l'objet de mesures exceptionelles et drastiques.
Les deux causes principales sont l'utilisation du pétrole, et celle du charbon.
Les conclusions que j'en tire sont que ces ressources DOIVENT être partagées le mieux du monde, elles permettent à l'homme de s'éviter un labeur de fou, étant donné la quantité d'energie disponible dans ces ressources, et en particulier le pétrole.
Une des hérésies de notre siècle est bien l'utilisation de la voiture individuelle (déplacer 1 tonne de métal sur des centaines de kilomètres juste pour satisfaire des besoins quaternaires est un comble!).
Un autre vecteur essentiel de cette dégradation est l'utilisation de l'avion dans ce même cadre. Le tourisme!! Créé pour satisfaire très temporairement des pauvres gens exploités dans des taches répétitives heure après heure, jour après jour, sans aucune perspective de sortie de ce processus sordide, si ce n'est le chomage, et la disparition à court terme des ressources nécessaires à sa propre vie, ou pire pour les individus ayant une famille, la disparition des ressources nécessaires à leurs enfants.\\
\begin{flushleft}
$ \Rightarrow $ Pas de voiture individuelle\\
$ \Rightarrow $ Recherche de l'air le plus pur possible\\
$ \Rightarrow $ Partage des ressources en pétrole\\
$ \Rightarrow $ Pas d'utilisation de l'avion\\
$ \Rightarrow $ Ne pas avoir peur de demander\\
$ \Rightarrow $ Réaliser que les gens prêtent facilement (sauf les amis... A expliquer)\\
\end{flushleft}

Les points positifs:

\subsubsection{L'eau}
Pour la conquête de l'eau la tâche commence à se compliquer un tant soit peu :-)
Il n'est pas si facile d'être autonome de ce point de vue. C'est le moment ou il faut commencer à partager la ressource, et réduire sa consommation au maximum afin de la préserver. Un puit et une éolienne de pompage constituent un minimum, mais on peut considérer que l'eau courante, si elle est correctement utilisée reste une option plus raisonable afin d'avoir une apparence descente, et envisager la vie en collectivité de façon plus agréable.
La vie sur un bateau et l'utilisation de sanitaires collectifs présentent beaucoup d'intérêts de ce point de vue, et comme je l'ai expérimenté pendant 2 ans et demi, je sais que c'est une solution tout à fait envisageable.\\
\begin{flushleft}
$ \Rightarrow $ Pas de voiture individuelle\\
$ \Rightarrow $ Pas de voiture individuelle\\
$ \Rightarrow $ Pas de voiture individuelle\\
\end{flushleft}

Les points positifs:


\subsubsection{La terre}
La terre de nos jours est exploitée, c'est à dire qu'elle ne sert plus du tout à subvenir aux besoins primaires, mais a été dovoyée, afin de proposer des besoins tertiaires et enrichir le plus rapidement possible un certain nombre de personnes "possédant".
Le sujet est très bien décrit dans un reportage Netflix: "The magic pill". Il serait heureux de trouver un document de référence sur l'étude de Timothy Noakes, afin de détailler son analyse et ses conclusions, mais bien entendu, comme souvent, le virage néfaste vient de l'Oncle Sam et de l'un de ces Anges de la mort:
La chaine alimentaire, telle qu'elle existait, et ce pendant des millénaires , ie l'homme chassant des bestioles, et se nourissant de protéines et de gras, ainsi que de fibres de la cueillette, a été "réinventée" par certains groupes d'intérêts pour faire des animaux d'élevage et de l'être humain, des consommateurs de céréales (et donc de sucre), ce qui a des conséquences énormes sur notre pauvre organisme, qui ne comprend plus rien et n'est pas prévu pour fonctionner de cette façon.
Ce constat explique en grande partie les "nouvelles" causes de mortalité (problèmes cardiovasculaires, cancer et diabète), qui sont complètement différentes des causes qui avaient pre existé, telle que les épidémies (propagées par les colonisations), les accidents et les guerres. 
Si l'on remonte encore, avant l'époque des grandes colonisations, et si l'on "évite" les conflits, on en revient aux causes "naturelles", c'est à dire au décès dûs à la vieillesse (le corps humain n'étant pas prévu pour être éternel).\\

\begin{flushleft}
$ \Rightarrow $ Pas de voiture individuelle\\
$ \Rightarrow $ Pas de voiture individuelle\\
$ \Rightarrow $ Pas de voiture individuelle\\
\end{flushleft}

Les points positifs:

\subsubsection{Le feu}
Par le feu, nous entendons l'abri, ou le confort minimal afin de se protéger des éléments, et des bactéries via la cuisson des aliments.
Pour cela, les besoins sont très limités, et ont été décrit dans mon analyse lorsque j'étais à Hong Kong. Cette analyse, que j'avais à l'époque intitulée "Climate Camp" était le resultat d'une introspection véritable et une quête de sens, au moment même ou je bénéficier de "tout", c'est à dire que potentiellement je pouvais accéder à tous les biens matériels, à tous les besoins décrits ici, et même à tout ceux que j'aurais pu oublier dans cette liste.
C'est l'origine de ce document en quelques sortes, et la conclusion que, bien que j'ai pris plusieurs décisions dans ce sens les dernières années, je ne suis pas parvenu à  l'objectif que je m'étais fixé (J'ai longtemps cru que c'était le cas, mais il s'est avéré que je me trompais).
Le fait est que je me suis laissé rattraper par des besoins tertiaires ou quaternaires (bien que j'en ai bientôt fini avec ces derniers), et cela, du fait que je n'ai pas trouvé de "collaborateurs", ou de sujet suffisament conscient des réalités décrites dans ce manifeste, pour parvenir à satisfaire mes besoins secondaires, et en particulier le partage.  
Bien que ces besoins primaires soit potentiellement accessibles individuellement, la tâche semble toutefois extrêmement plus difficile qu'au sein d'une communauté.
C'est pourquoi il est nécessaires de s'attarder sur les besoins seondaires, et les potentialités qui en découlent.\\
\begin{flushleft}
$ \Rightarrow $ Pas de voiture individuelle\\
$ \Rightarrow $ Cf Hong Kong climate camp\\
$ \Rightarrow $ Pas de voiture individuelle\\
$ \Rightarrow $ Pas de voiture individuelle\\
\end{flushleft}

Les points positifs:
\begin{flushleft}
$ \Rightarrow $ Couchsurfing\\
\end{flushleft}

\subsection{ Les besoins secondaires}

\subsubsection{L'entraide et le partage}
Est-ce un besoin primaire, ou, l'hommme est-il un animal "social".

Un exemple extra-ordinaire est la navigation en équipage.
Il est un constat qu'il est facile de faire, énormément de choses deviennent possibles lorsque l'on est deux, qui sont complètement impossibles lorsque l'on est seul. Malheureusement l'hommme ne sait plus partager. On ne parle pas ici d'actes gratuits pour la seule satisfaction d'un autre individu, mais bien de la définition d'un objectif collectif, et pour le bien de ce collectif.
La pricipale force qui retient les être humains de collaborer est la peur que leurs actions communes ne soient récupérées à terme par un seul individu, ou un petit groupe d'individus, et c'est là bien sûr le grand danger. 
En effet tous les gens ne sont pas égaux, et la nature les gâtes de diverses qualités qui sont généralement bien réparties au sein de la population, c'est ma conviction intime.
Les handicaps lourds (psychologiques ou physiques) ou les personnes en fin de vie étant bien évidemment dans une situation très particulière du point de vue de leur contribution au collectif.
Toutefois leur proportion très minoritaire en nombre au sein de la population devrait permettre de leur allouer des ressources particulières sans affecter la vie des autres individus pour peu qu'ils ne soient pas isolés.
Certains sont plus doués pour les activités intellectuelles, et d'autre pour les activités physiques. Il me semble qu'en prenant un échatillon suffisament représentatif il doit être possible de mettre à contribution tout le monde pour la poursuite des objectifs définis.
Pour cela il est impératif que tout le monde reconnaisse le principe de: "On a toujours besoin d'un plus petit que soi".\\

\begin{flushleft}
$ \Rightarrow $ Définir précisement ce qui relève de la propriété individuelle\\
$ \Rightarrow $ Définir précisement ce qui relève de la propriété collective\\
$ \Rightarrow $ Communiquer de façon COMPLETEMENT transparente et formaliser systématiquement\\
$ \Rightarrow $ Définir le cadre de l'organisation telle que draftée dans la gestion de projet\\
$ \Rightarrow $ Convaincre les plus forts de réduire leurs prérogatives afin de préserver la cohésion (dans leur propre intérêt)\\
\end{flushleft}

Les points positifs:


\subsubsection{La tendresse}
Il n'est pas possible d'obliger l'être humain à la tendresse, celle-ci doit lui venir naturellement, il convient donc d'identifier les conditions qui permettront ce partage de chaleur humaine (bien que celle-ci soit nécessaire, il est bien plus facile d'en recevoir que d'en donner, toutefois cela va dans les deux sens).
Cette tendresse ne peut devenir naturelle chez l'homme que s'il a par ailleurs assuré ses besoins fondamentaux, et qu'il a banni toute sorte de besoins factices, au sein d'une communauté (et non d'un groupe avec dogme, symbole, charia, mais nous y reviendrons). 
Il convient ici de noter que cette tendresse nous vient naturellement des animaux qui nous entourent, la psychologie de ces animaux n'a pas encore été dévoyée, et il est très commun de nos jours que l'être humain se raccroche à ses animaux de compagnie pour ce besoin.
Ses semblables sont tellement pervertis par le matérialisme qu'ils se refusent maintenant à tout acte "gratuit". Ce qui a conduit à l'un des plus grands désastres des temps modernes, la prostitution.

\subsection{Les besoins tertiaires}

\subsubsection{Le plaisir des sens}
\underline{Le gout}\\
\underline{L'ouie}\\
\underline{Le toucher}\\
\underline{La vue}\\
\underline{L'odorat}\\
https://www.corps.dufouraubin.com/sens/sens.htm
Ceux-ci pourraient correspondre à tous les besoins tertiaires ou pas...

\subsubsection{L'activité intellectuelle}
\underline{L'ingénierie}\\
L'ingénierie nous permet de subvenir à une grande quantité de nos besoins, celle-ci est fondamentale afin de construire une maison, mettre à disposition de l'eau, etc. Toutefois elle doit se limiter aux besoins de type primaire, secondaire ou tertiaire, et en aucun cas d'être motivée par l'un ou l'autre des vices de l'homme. La paresse, la haine, la convoitise, la jalousie, etc. Ces vices sont sans doute issus de l'intéraction avec d'autres hommes, et l'esprit de compétition. Cet esprit de compétition et le refus du partage conduit à la perte de l'être humain.\\
\underline{L'art}\\
\underline{La science}\\
\underline{La philosophie}\\
\subsubsection{Le rêve}
L'astronomie, la contemplation de la nature. Les voyages, l'émerveillement disparu avec l'éloignement de la nature. Réservé de nos jours aux "congés payés".
Lorsque j'étais à Hong Kong, perdu pour la vie, j'ai pensé à une liste de choses qui la représentait:
\begin{itemize}
\item
Un film drole
\item
Une partie de pèche
\item
Une après-midi à la barre

\end{itemize}
\subsubsection{L'eau}

\subsection{Les besoins quaternaires (ou artificiels) physiques}
Les besoins quaternaires découlent le plus souvent de l'aliénation de l'homme par d'autres hommes, et de la perte de perspectives et de "vision" du futur pour ces êtres humains (trop humains), écrasés par un système qu'ils ne comprennent plus, qui est en violation la plus complète avec la base de la vie sur terre, mais auquel ils sont tenus de contribuer coute que coute sous peine de se voire mettre au rebus.
J'ai moi-même expérimenté tous les affres de cette consommation de "produits" afin d'anesthésier les douleurs psychiques auxquelles je faisais face.
Certains de ces produits permettent artificiellement de "s'anesthesier", ou plutot son cerveau, afin d'alléger la souffrance associée.
\subsubsection{L'adrénaline}
Toutes personnes soumise à un stress, qu'il soit important ou pas, produit de l'adrénaline. Cette adrénaline se diffuse dans le corps, et permet un certain nombre de choses. Une augmentation de l'attention, des réflèxes et de l'énergie disponible.
Le métabolisme d'un sujet soumis à un excès d'adrénaline permanent ou quasi permanent, s'adapte à cette production/réutilisation et celle-ci devient "naturelle", et se traduit par un besoin lorsque celle-ci vient à manquer.
Le sujet en vient alors à avoir un comportement "risqué", afin de subvenir à ce besoin. J'ai longtemps moi-même été dépendant de cette substance, ce qui m'a amené à accepter des situations que beaucoup d'autres auraient refusé.
Note: Il parait que les gens qui ont un pouls très bas tendent à prendre des risques.
Le besoin d'adrénaline vient aussi du stress qui entoure la personne, ou celui qu'elle se met à elle-même.
Identification du cercle vicieux:
\subsubsection{Le sucre}
La drogue la plus courante qui existe. Quels facteurs ont déclenché ce besoin? Ou alors le besoin a-t-il été créé de toute pièce, mais dans quel but?
Il s'agit ici d'utiliser la substance pour vendre des produits très rentables comme le Coke, et tout un tas d'autres dérivés.
Rentabiliser la dépendance que provoque le sucre a été exploité au maximum, reste à savoir pourquoi et comment il est si bon marché. 
Here we go: \url{https://en.wikipedia.org/wiki/Tim_Noakes}
Identification du cercle vicieux:
\subsubsection{La nicotine}
Une substance très vicieuse. Pas de "hangover", pas de perte de concentration (pas d'augmentation non plus contrairement à ce que le fumeur peut en dire),
pas de... Juste un léger sentiment de manque physique, une dépendance psychologique monstrueuse, et des poumons fatigués.
Cette drogue génère des profits considérables pour les producteurs et les états qui les taxes de façon délirantes sous couvert de protéger la population.
Identification du cercle vicieux:
\subsubsection{L'alcool}
Sans aucun doute la seule drogue "dure" dans ma liste. Son sevrage ne provoque pas de manque physique, elle correspond plus à une amie (qui devient au fur et à mesure la seule), et la aussi la dépendance psychologique est énorme, mais en soit complètement différente de la dépendance à la nicotine.
Ici les effets sur le physique sont très violents. La personne perd pied, au sens litéral, et connaissance. La consommation desinhibe complètement le sujet, et lui font ressortir toutes les frustrations et joies intérieures. Cela peut se traduire par un comportement sans aucune espèce de conscience "morale", et une hyper sensibilité.
Les conséquences physiques se font ressentir pendant de longues heures, tant le corps a du mal à assimiler le produit.
La consommation d'alcool est due à un manque de reconnaissance, autant par les autres que par soi-même.
Substitut au rêve?
Identification du cercle vicieux:
\subsubsection{Les opiacés}
Pour ce qui est des opiacés, je distingue plusieurs nuances (nuances c'est important pour éviter de faire des catégories, et mettre es choses dans des sacs).
Identification du cercle vicieux:
Substitut au toucher? 

\subsubsection{La pornographie}

On en vient à se demander dans quelle mesure la consommation des produits quaternaires provient du manque des besoins primaires...
De plus il y a sans doute un autre facteur qui entre en jeu ici, c'est le concept de morale. Plus les gens se sentent coupables, plus ils doivent trouver un moyen d'oublier leur pêchers. C'est pourquoi il est si facile de maintenir le troupeau dans la voie qu'on lui a fixé.
Identification du cercle vicieux:
Substitut au sexe? 

\subsection{Les pêchers capitaux (source et conséquence des besoins artificiels)}
Les pêchers capitaux n'ont pas été définis pour rien, ils sont l'émanation de la sagesse des ages.\\
Ou alors ils ont été définis pour rendre les peuples dociles pendant que les maitres profitent.\\
\subsubsection{L'avarice}
\subsubsection{Les pêchers capitaux}
\subsubsection{Les pêchers capitaux}
\subsubsection{Les pêchers capitaux}

\subsection{Les droits de l'homme}
\subsubsection{Le respect}
\subsubsection{Le respect}
\subsubsection{Le respect}
\subsubsection{Le respect}
\subsubsection{Le respect}

\end{document}
